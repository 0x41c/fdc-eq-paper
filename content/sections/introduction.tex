% Introduction should not have definitions.
The finite difference operation is an important tool used in numerical analysis and calculus. Its main pupose is to provide a numerical approximation to the derivative, and it does that well. One will find that with fine tuning of coefficients of the difference operation The most well known example of its use comes from the definition of the derivative:
\begin{equation}
  \frac{dx}{dy}=\lim_{\Delta{x}\to{0}}\frac{f(x+\Delta{x})-f(x)}{\Delta{y}}
  \label{eq:def:derivative}
\end{equation}
The forward difference operation appears in expression \hyperref[eq:def:derivative]{(1)} in the form of $f(x+d)-f(x)$ where d is equivalent to $\Delta{x}$ when discarding the derivative acting on it. The formal definition of the forward difference in a general form is:
\begin{equation}
  \Delta^{n}_{h}\left[f\right]\left(x\right)=\sum^{n}_{k=0}\left(-1\right)^{n-k}\binom{n}{k}f(x+hk)
\end{equation}
The motivation for finding an alternative definition of the difference operation stems from an analysis on the resulting functions given by the operation. The outputs of the difference operation and the derivative relate by the following assertion:
\begin{equation}
  \label{eq:asrt:ordering-df-dv}
  \frac{1}{n!}f^{\left(n\right)}h^n\leq\Delta^n_h\left[f\right]\left(x\right)
\end{equation}
% TODO: Highlight the goal of proof & align it with the motivation for the paper.
In the following proof, we'll establish this relationship by deriving a general equation for the finite difference in terms of the derivative. Not all cases of the difference will be proven in this paper. The backward and central differences can be considered as an exercise for the reader.

% At the end of the proof we'll have determined that the following equation models the finite forward difference:
% \begin{equation}
%     \Delta^{n}_{h}\left[f\right]\left(x\right)=\sum^{1}_{k_1=1}\underbrace{\dots}_{n-2}\sum^{\deg{f}}_{k_n=1}\frac{h^{k_1+\dots+k_n}}{\prod^n_{i=1}k_{i}!}f^{\left(k_1+\dots+k_n\right)}\left(x\right)
% \end{equation}